\section{Use the velocity Verlet algorithm to study liquid argon}

\Question For this problem set, you should implement the velocity Verlet algorithm and use
it to study liquid argon. You should use the
velocity Verlet algorithm for your simulations. It is self-starting, making it slightly
easier to use than the simple Verlet algorithm in Verlet's original paper.

This is a study of a collection of classical particles interacting through a Lennard--Jones
potential, a central force potential. The evolution of this system according to Newton's
Laws represents a microcanonical ensemble, since the energy is fixed. The temperature of the
system can be found from the average kinetic energy.

In writing your program, there are a number of issues to consider:

\begin{enumerate}
    \item The system should be put in a box of size $L$ on a side. The desired density and
          input number of particles will determine the box size. Periodic boundary
          conditions should be used.
    \item The initial state of the system must be chosen. You can easily choose the
          velocities $\vec{v}$ so that each component is uniformly distributed between
          $-v_\text{max}$ and $v_\text{max}$. After choosing the velocities for each
          particle, shift all velocities so that the velocity of the center of mass is zero.
          Alternatively, you can place all the particles initially at rest and then they
          will acquire velocities as the systems thermalizes.

          Putting the particles at random positions inside the box will generally result in
          very large potential energies. A few relaxation steps, moving each particle
          individually a small distance in the direction of the force acting on it, will
          dramatically decrease the potential energy. You can also arrange the particles in
          a somewhat regular pattern, to avoid any two particles being too close together.
          The important point is that the initial configuration does not matter, provided
          there are no particles close enough to each other that the $\frac{ 1 }{ r^{12} }$
          term in the potential becomes large.
    \item Once chosen, the total energy will remain fixed, provided your algorithm is
          working correctly. The temperature will be set, after thermalization, by the
          resulting balance between kinetic and potential energy.
    \item The energy and force for particle $a$ due to particle $b$ should be calculated
          using the ``nearest image'' of particle $b$.
    \item While integrating the equations of motion, some particles will move out of the
          box. By a shift of $L$ in the appropriate coordinate, the particle can be brought
          back into the box.
\end{enumerate}

You should check your program by studying whether it conserves energy and whether the
particles motion is reversible, i.e., whether the system returns to an earlier state when
the velocities are reversed and the molecular-dynamics restarted.

Once you believe your program is working, you should reproduce the values for pressure and
internal energy found by Verlet for $\rho = 0.75$ and $T = 1.069$. You can fine-tune the
temperature by small rescaling of the velocities or by moving particles a small amount in
the direction of the force they are experiencing, i.e., move them downhill. (Remember that
you need to run the molecular dynamics for some time after such an adjustment, to let the
system come back into thermal equilibrium.)

To get reasonable accuracy for $T$ and $P$, runs of a few thousand molecular dynamics steps
should be sufficient, after thermalization. You do not need to reproduce the temperature to
precisely $1.069$.

Some additional features you should have in your program and results you should produce are:

\begin{enumerate}
    \item You should be able to write the position and velocity of all the particles to a
          file, at intervals of your choosing. You should also be able to read this file in
          and continue your simulation. This will allow you to add more molecular dynamics
          time to a simulation, without re-starting from an unthermalized state.
    \item Record in a file the measurements necessary to determine the pressure, via the
          virial theorem, every $10$ molecular dynamics steps. We will use these later to
          determine the statistical errors on your measurements.
    \item Determine the velocity distribution of the particles by recording them after every
          $10$ molecular dynamics steps. Does it agree with the expected Maxwell distribution?
\end{enumerate}

\Answer 1

