\section{Problem \thesection}\label{sec:p1}

In this example, we will look at the propagation of finite precision errors in
multiplication. We start by truncating the input numbers to a fixed precision, given by
ndig, and compare this to the result without truncating. To keep things relatively simple,
we start from random numbers distributed uniformly between $0$ and $1$. We then shift these,
to keep the result from getting either too large or too small.

In the example code below, to round a floating point number, it is first decomposed by
\texttt{math.frexp(x)} and the mantissa is rounded to a desired number of digits. Then the
rounded mantissa is multipled by the exponent to give a rounded version of the floating
point number.

\begin{minted}[frame=single]{python}
import numpy as np
import matplotlib.pyplot as plt
import math

#  The number of random numbers we will multiply together
nrand = 10001

#  The number of digits we truncate the operand to
ndig = 6

#  Start with x and multiply this by nrand random numbers
x = np.random.rand()
r = np.random.rand(nrand) + 0.542

#  Find the rounded value for x, xr
#  Need intermediate decomposition, xd
xd = math.frexp(x)
xr = round(xd[0],ndig) * 2**xd[1]

#  xt is truncated value
xt = float(int(x * 10**ndig)) * 10**-ndig

print('n\t\t\t','Full\t\t\tRounded\t\t\tTruncated')
print('0\t\t\t',x,'\t',xr,'\t\t',xt)

for n in np.arange(nrand):

    x = x * r[n]

    #  Round random number and round result
    rd = math.frexp(r[n])
    xr = xr * round(rd[0],ndig) * 2**rd[1]
    xrd = math.frexp(xr)
    xr = round(xrd[0],ndig) * 2**xrd[1]

    #  Truncate
    xt = xt * float(int(r[n] * 10**ndig)) * 10**-ndig

    if n in [ 10, 100, 1000, 10000, 100000 ]:
        print('\n%d\tvalue\t\t%14.8e\t\t%14.8e\t\t%14.8e'% (n, x, xr ,xt))
        print('\tfrac diff\t%3.1f\t\t\t%5.2e\t\t%5.2e' % (x/x, (x-xr)/x,(x-xt)/x))
\end{minted}

Notice how the deviation for the truncated case is growing $10$ times worse with every $10$
times increase in floating point operations.

For this problem, you should start from this code and adapt it to be run many times in a
loop (hundreds of times). For each time through the loop, you should store the fractional
differences for both rounding and truncation after $100$, $1000$ and $10000$ multiplications.
Plot a histogram of the deviations for both cases and the three different numbers of
multiplications. Plot the average difference versus the number of multiplies for both the
rounding and truncation cases to see how it scales with the number of multiplies.
